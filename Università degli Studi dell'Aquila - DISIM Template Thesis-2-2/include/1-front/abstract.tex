\thispagestyle{plain}			% Supress header 
\setlength{\parskip}{0pt plus 1.0pt}
\section*{Abstract}
Pull request messages describe the changes made in software development, providing essential context for understanding code modifications and aiding in the comprehension of software evolution. Developers tend to manually draft these messages, but research has introduced several strategies to automate their generation, including template-based approaches, model learning, and information retrieval. This thesis proposes a novel approach for generating pull request messages using a large language model (LLM). The LLM, Ollama, is employed with one-shot, few-shot, and other prompt configurations to generate either the title or the body message of pull requests, based on the metadata and code changes. Unlike retrieval-based techniques, the proposed approach directly leverages the dataset to generate accurate and coherent messages without requiring external retrieval mechanisms. The evaluation of this method uses BLEU, METEOR, and cosine similarity to assess the quality of the generated pull request messages. Multiple experiments are conducted across different prompt configurations to explore the best approach for generating effective pull request descriptions.

% KEYWORDS (MAXIMUM 10 WORDS)
\vfill


\thispagestyle{empty}
\mbox{}
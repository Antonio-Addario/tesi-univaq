\chapter{Introduction}
"Enabled by the pull-based development model, developers can easily contribute to a project through pull requests (PRs). When creating a PR, developers can add a free-form description to describe what changes are made in this PR and/or why. Such a description is helpful for reviewers and other developers to gain a quick understanding of the PR without touching the details and may reduce the possibility of the PR being ignored or rejected."\citet{8952330}. This seemingly simple process can be complex for several reasons: descriptions may lack essential information, be unclear, or inconsistent with the changes made. Additionally, "Although PRs help developers improve the development efficiency, some developers usually ignore writing the descriptions for PRs."\citet{fang_prhan_2022}. The quality of these messages directly impacts the understanding of the code and the efficiency of collaborative work, especially in large projects with many developers. One possible solution to this problem is to automate the generation of pull request titles and descriptions. "The description of an item plays a pivotal role in providing concise and informative summaries to captivate potential viewers and is essential for recommendation systems.  Traditionally, such descriptions were obtained through manual web scraping techniques, which are time-consuming and susceptible to data inconsistencies. In recent years, Large Language Models (LLMs), such as GPT-3.5, and open source LLMs like Alpaca have emerged as powerful tools for natural language processing tasks."\citet{10.1145/3604915.3610647}. Using large language models (LLMs) allows you to automate the process, speeding up the work of contributors and ensuring greater consistency and completeness of information. The basic idea is to leverage pull request metadata, such as code changes and associated issues, to automatically generate high-quality descriptive messages. Furthermore, such a tool should be flexible and customizable, allowing it to adapt to different types of projects and user preferences, for example by highlighting or not certain information. This thesis focuses on the study and experimentation of the use of LLMs for the automatic generation of pull request titles and descriptions. The main goal is to explore the potential of this technology and lay the foundations for the future development of a dedicated tool. The chapter \ref{chap:background} will present the fundamental concepts related to the project, including an overview of the language models, pull requests and metadata involved. The chapter \ref{chap:pipeline} will describe the design and implementation process, illustrating the organizational choices, the technical solutions adopted and the functions developed to address specific problems. The chapter \ref{chap:validation} will present the experiments conducted, introducing the research questions that guide the study, the methodology adopted, and the evaluation metrics used to assess the generated outputs. Additionally, it will provide a detailed analysis of the adopted configurations, including the type of LLM, configuration parameters, and prompt structuring for each experiment.
 In the chapter \ref{chap:results} the results will be illustrated through graphs, which allow to visualize the distribution of the evaluation metrics for each experiment. A total of eight experiments were conducted, divided into four configurations for generating descriptive titles and four for generating body messages of pull requests. The results highlight how the generation of titles obtained higher scores than the body messages. This result seems consistent with the very nature of titles, which tend to be more "deterministic", while body messages vary significantly based on the user's style and the project context. The results obtained confirm the potential of LLMs to support software development processes, opening interesting perspectives for the automation of activities that, until recently, required direct human intervention.